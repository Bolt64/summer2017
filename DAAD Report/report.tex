\documentclass[12pt, notitlepage]{article}
\usepackage[utf8]{inputenc}

\usepackage[nottoc]{tocbibind}

\usepackage[left=3cm,right=3cm,top=3cm,bottom=3cm]{geometry}

\title{WISE 2017 Experience Report}
\author{Sayantan Khan \\ \texttt{sayantangkhan@gmail.com} \\ Indian Institute of Science}
\date{August 2017}

\usepackage{gentium}

\makeatletter

\makeatother

\begin{document}
\maketitle

It had been a long held belief in the mathematical community that German mathematics
never really recovered from the exodus of a large fraction of the German mathematics
community in the 1930s. Since then, the centre of mathematical activity has been the United
States, in particular the communities in Princeton and the universities in Boston,
where a lot of the exiled mathematicians went. The situation now, however, is taking
a turn for the better: a lot of mathematical activity has started attracting attention
of the worldwide community; activity by the homegrown mathematicians of Germany, who
are making Germany the centre of world's mathematics once again.

As a DAAD-WISE scholar, I had a first hand opportunity to be a witness to the flurry
of mathematical activity universities in Germany are a host to. In particular, I spent
the months of May, June, and July as a part of the topology group at the University of
Münster. I attended a few graduate courses at the university, along with participating
in the group's work, i.e. seminars and problem discussions.

I was in for a bit of a shock when I started attending the courses. Throughout all my time
at my institution, I had aced through the graduate courses, and as a result, had grown rather
complacent. But the courses at the University of Münster were a completely different ball game.
They were fairly dense, with a lot of material packed into them, and the problem sets were rather
hard. It took me a few weeks to get up to speed, and pace myself with the class, but nevertheless,
it was a valuable lesson, and I'm glad it got hammered into me in time.

The seminars were another important takeaway lesson. They were comprehensive, well organized, and most
importantly, they required each person in the group to participate, i.e. give a lecture at least once,
if not more than once. Again, this was in stark contrast with the seminar style at my institution, which
usually involve a small subset of the entire group giving the lectures, whereas the rest are passive
spectators. Consequently, along with another classmate who was also a WISE fellow, I am planning to
organize a German style seminar on Algebraic Geometry this semester. If this style of seminar
catches on in the department, that will be the biggest takeaway from my summer in Germany.

As for the specifics of my academic learning in Germany, I learnt a fair amount of algebraic topology,
especially the theory of stable homotopy, representable functors, and spectra. It would have been difficult
to learn this material in India, given that there aren't too many algebraic topologists in India. Furthermore,
I now know what I want to specialize in when I join graduate school.

One issue I did have though was that my lack of German speaking skills excluded me from a lot of public
lectures, and classes I could have attended, had they been in English. This however, did give me a sufficiently
strong incentive to try and learn as much German as I could; I managed to learn enough German to read mathematics
textbooks.

As for learning on the personal front, the most important thing I learnt was how to cook on a budget.
I had to plan out my entire week's eating in advance to make sure I had enough raw materials to cook what
I needed. I also got to socialize with the other students at the university, especially the Erasmus fellows.
It was quite enlightening, meeting such a diverse group of people.

To conclude, I believe this summer was quite enriching, and I learnt a fair amount of mathematics,
but more importantly, I got a glimpse of the German way of doing mathematics, and I hope some of the
famed German work ethic rubbed off on me.

\end{document}
