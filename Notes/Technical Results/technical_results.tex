\documentclass[12pt, notitlepage]{article}
\usepackage[utf8]{inputenc}
\usepackage{graphicx}
\graphicspath{ {images/} }

\usepackage[english]{babel}
\usepackage[nottoc]{tocbibind}

\usepackage{hyperref}
\usepackage[left=3cm,right=3cm,top=2cm,bottom=2cm]{geometry}

\usepackage{bbm}

\usepackage{amsmath}
\usepackage{amsthm}
\usepackage{amsfonts}
\usepackage{amssymb}
\newtheorem{thm}{Theorem}[section]
\newtheorem{lem}[thm]{Lemma}
\newtheorem{prop}[thm]{Proposition}
\newtheorem{cor}[thm]{Corollary}
\newtheorem{conj}[thm]{Conjecture}
\newtheorem{exmp}[thm]{Example}
\newtheorem{remark}[thm]{Remark}

\usepackage{mathtools}
\usepackage{tikz-cd}

\usepackage{xcolor}

\theoremstyle{definition}
\newtheorem{defn}{Definition}[section]

\title{Proofs of some technical results}
\author{Sayantan Khan}
\date{July 2017}

\usepackage{gentium}

\newcommand{\cat}[1]{\mathrm{#1}}
\newcommand{\cohomtheorie}{{h}^{\ast}}
\newcommand{\calz}{\mathcal{Z}}

\makeatletter
\newcommand{\colim@}[2]{%
  \vtop{\m@th\ialign{##\cr
    \hfil$#1\operator@font colim$\hfil\cr
    \noalign{\nointerlineskip\kern1.5\ex@}#2\cr
    \noalign{\nointerlineskip\kern-\ex@}\cr}}%
}
\newcommand{\colim}{%
  \mathop{\mathpalette\colim@{\rightarrowfill@\textstyle}}\nmlimits@
}
\makeatother

\begin{document}
\maketitle

\tableofcontents

\newpage

\section{Blakers-Massey Theorem}
\label{sec:blak-mass-theor}

Fill in later

\section{Comparison theorem for cohomology theories}
\label{sec:comp-theor-cohom}

Fill in later

\section{Brown's representability theorem}
\label{sec:browns-repr-theor}

In this section, we shall see that all reduced cohomology theories that satisfy the wedge sum
($\mathrm{DV}$) axiom are representable functors, i.e. they are naturally isomorphic to the hom
functor in the homotopy category $\cat{hCW}_{\ast}$. In particular, for a given reduced cohomology
theory $\cohomtheorie$, we'll construct a sequence of spaces $\calz(n)$, which we'll call a
spectrum, such that $h^n(X)$ is naturally isomorphic to $\left[X, \calz(n)\right]$.

\subsection{Spectra and cohomology theories}
\label{sec:spectra-cohom-theor}

\begin{defn}[$\Omega$-Prespectrum]
  A prespectrum is a $\mathbb{Z}$ indexed sequence of pointed spaces $\calz(n)$ together with
  structure maps $\sigma_n: \Sigma \calz(n) \to \calz(n+1)$. If the adjoints of the structure maps,
  i.e. the maps $\widetilde{\sigma}_n: \calz(n) \to \Omega \calz(n+1)$ are homotopy equivalences,
  then the prespectrum is called an $\Omega$-prespectrum.
\end{defn}

\begin{prop}
  Given a $\Omega$-prespectrum $\calz$, one can define the following functor.
  \begin{align*}
    h^n(X; \calz) = \left[X, \calz(n)\right]
  \end{align*}
  This is a contravariant functor which satisfies the homotopy invariance $(\mathrm{H})$, suspension
  $(\mathrm{S})$, exactness $(\mathrm{E})$, and the wedge sum $(\mathrm{DV})$ axiom. It is therefore
  a reduced cohomology theory.
\end{prop}

\begin{proof}
  We'll deal with the axioms one at a time.
  \begin{description}
  \item[Homotopy invariance (H):] This is obvious, because we are looking at homotopy classes of
    maps.
  \item[Suspension (S):] We need to show there is a natural isomorphism from $h^n(X)$ to
    $h^{n+1}(\Sigma X)$. Note that the adjoint of the structure maps are homotopy equivalences.  We
    therefore have a natural isomorphism.
    \begin{align*}
      \left[ X, \calz(n) \right] \cong \left[ X, \Omega \calz(n+1) \right]
    \end{align*}
    On the other hand, since $\Sigma$ are $\Omega$ are adjoints, we have the following natural
    isomorphism.
    \begin{align*}
      \left[ X, \Omega \calz(n+1) \right] \cong \left[ \Sigma X, \calz(n+1) \right]
    \end{align*}
    Composing the two natural isomorphisms, we get our required isomorphism.
  \item[Exactness (E):] We need to show for any cofibration $i: A \hookrightarrow X$, the following
    sequence is exact.
    \begin{align*}
      h^n(A) \leftarrow h^n(X) \leftarrow h^n\left( X/A \right)
    \end{align*}
    Using the cofibre sequence, we get that following sequence is exact.
    \begin{align*}
      [\Sigma A, \calz(n)] \leftarrow [\Sigma X, \calz(n)] \leftarrow \left[\Sigma \left(X/A\right), \calz(n) \right]
    \end{align*}
  \item[Wedge sum (DV):] The functor $[\cdot, \calz(n)]$ satisfies $(\mathrm{DV})$ axiom. This is
    fairly easy to check. That means $h^{\ast}$ satisfies $(\mathrm{DV})$ axiom.
  \end{description}
\end{proof}

{\color{red} To construct easy examples of spectra, one needs to check that filtered colimits
  commute with the loop space functor, at least for nice enough spaces.}

\subsection{Proof of Brown's representability theorem}
\label{sec:proof-browns-repr}

In the previous section, we saw that if we are given an $\Omega$-prespectrum, we can construct a
reduced cohomology theory using the prespectrum. Brown's representability theorem is the converse of
the previous theorem, i.e. given a reduced cohomology theory which satisfies the $(\mathrm{DV})$
axiom, it can be represented by an $\Omega$-prespectrum, which is unique up to homotopy. This
theorem is fairly technical, and will require the use of the theorem on Milnor exact sequence
(theorem \ref{thm-milnor}).

\begin{thm}[Brown's representability theorem]
  Let $h^{\ast}$ be a reduced cohomology theory satisfying the $(\mathrm{DV})$ axiom. Then there is
  an $\Omega$-prespectrum $\calz$ such that $h^{n}$ is naturally isomorphic to
  $\left[\cdot, \calz(n) \right]$. Furthermore, this $\Omega$-prespectrum is unique up to homotopy.
\end{thm}

\begin{proof}
  The proof will have two main parts. The first part will involve constructing the spaces $\calz(n)$
  for each $n$ such that there is a natural isomorphism from $h^n(X)$ to
  $\left[ X, \calz(n) \right]$ for all CW complexes $X$. The second part will involve constructing
  the structure maps from $\Sigma \calz(n) \to \calz(n+1)$.

  Fix an $n \in \mathbb{Z}$. We will construct the space $\calz(n)$ as a CW complex, using finite
  dimensional skeletons $\calz(n)_k$. For each $k$, we will also pick a cohomology class $c_n(k)$ in
  $h^n(\calz(n)_k)$ such that the map $d_n^{m}(k): \left[S^m, \calz(n)_k \right] \to h^n(S^m)$ is an
  isomorphism for $m < k$ and surjection for $m=k$.
  \begin{align*}
    d_n^{m}(k) &: \left[ S^m, \calz(n)_k\right] \to h^n(S^m) \\
    d_n^{m}(k) &: [f] \mapsto f^{\ast}(c_n(k))
  \end{align*}
  For $k=0$, we define $\calz(n)_0$ as follows.
  \begin{align*}
    \calz(n)_0 := \bigvee_{\alpha \in h^n(S^0)} S_{\alpha}^0
  \end{align*}
  The cohomology group of $\calz(n)_0$ is given by a direct product, since $h^n$ satisfies the
  $\mathrm{(DV)}$ axiom.
  \begin{align*}
    h^n(Z(n)_0) \cong \prod_{\alpha \in h^n(S^0)} h^n(S_{\alpha}^0)
  \end{align*}
  Pick the following element as $c_n(0)$.
  \begin{align*}
    c_n(0) := \prod_{\alpha \in h^n(S^0)} \alpha
  \end{align*}
  Since $k=0$, we only need to show that $d_n^0(0)$ is a surjection. Pick any $\alpha \in h^n(S^0)$.
  Corresponding to this $\alpha$, there's a copy of $S^0$ sitting inside $\calz(n)_0$. Let $f$ be
  the inclusion map of this copy of $S^0$ into $\calz(n)_0$. Then the induced map on cohomology is
  the projection map on the $\alpha$\textsuperscript{th} coordinate, since the cohomology theory
  satisfies the $(\mathrm{DV})$ axiom. Applying this induced map on $c_n(0)$, we see that in the
  $\alpha$\textsuperscript{th}, it has $\alpha$, because of the way we defined it.  This shows the
  map is surjective.
\end{proof}

\newpage

\appendix

\section{Definitions and notation}
\label{sec:definitions-notation}

\begin{defn}[Suspension of a pointed space]
  The suspension $\Sigma X$ of a pointed space $X$ is the smash product $S^1 \wedge X$.
\end{defn}

\begin{defn}[Loop space of a pointed space]
  The loop spaces $\Omega X$ of a pointed space $X$ is the set of all pointed maps from $S^1$ to $X$
  with the compact-open topology.
\end{defn}

\begin{defn}[$\lim^1$]
  Let $T$ be the category of towers of abelian groups, i.e. $\mathbb{N}$ indexed set of abelian
  groups $G_i$ with maps $f_i: G_i \to G_{i-1}$, and maps are set of arrows that make the whole
  thing commute {\color{red} (Check that this category has enough injectives)}. Then $\lim$ is a
  left exact functor from $T$ to $\cat{AbGrp}$ {\color{red} (This is easy to check)}, and we define
  $\lim^1$ to be the first right derived functor of $\lim$.
\end{defn}

\section{Some useful lemmas and theorems}
\label{sec:some-useful-lemmas}

\textbf{Note:} Although we state many of the lemmas here for $\cat{TOP}$, they are also true for
$\cat{TOP}_{\ast}$, and the proof is similar.

\begin{lem}
  If $i: A \hookrightarrow X$ is a cofibration (in the category $\cat{TOP}$), then the mapping cone
  $C(i)$ is homotopy equivalent to $X/A$.
\end{lem}

\begin{proof}
  We will first construct the maps to and from $C(i)$ to $X/A$. The maps from $C(i)$ to $X/A$ is the
  map that collapses the cone of $A$ to a point corresponding to $A$ in $X/A$. Now consider a map
  from $H: A \times I$ to $C(i)$, such that $H$ contracts $A$ to a point in $C(i)$, starting from
  the inclusion of $A$ in $X$. Let the map $j$ from $X$ to $C(i)$ be the inclusion map. Since $i$ is
  a cofibration, we can extend $H$ with the initial condition $j$ to a map $J: X \times I \to
  C(i)$. But $J(\cdot, 1)$ collapses $A$ to a point. That means it factors through a $X/A$. This
  gives us a map $k$ from $X/A$ to $C(i)$.

  The fact that these maps are homotopy inverses can be verified using the homotopy
  $J$. ({\color{red} Not sure of this. Verify later.})
\end{proof}

\begin{lem}
  In the category $\cat{TOP}$, the following sequence is h-coexact.
  \begin{align*}
    A \xrightarrow{f} B \xrightarrow{i} C(f)
  \end{align*}
  That means for any space $Z$, the following sequence of abelian groups is exact.
  \begin{align*}
    [A, Z] \leftarrow [B,Z] \leftarrow [C(f), Z]
  \end{align*}
\end{lem}

\begin{proof}
  If an element $[c] \in [B,Z]$ goes to $0$ in $[A, Z]$, that means $c \circ f: A \to Z$ is
  nullhomotopic, where $c$ is a representative of $[c]$. But that means there is some function
  $d \in C(f)$ such that $c = d \circ i$. This shows the exactness of the sequence.
\end{proof}

\begin{lem}
  If $K$ is a compact space, let $A_i$ be a sequence of spaces where points are closed, and $A$ is
  the colimit of the following diagram:
  \begin{align*}
    A_0 \hookrightarrow A_1 \hookrightarrow A_2 \hookrightarrow \cdots
  \end{align*}
  where all the embeddings are closed, then a map from $K$ to $A$ factors finitely through some
  $A_i$.
\end{lem}

\begin{proof}
  Let $J = f(K)$ be the compact image of $K$ in $A$. For each set $A_i \setminus A_{i-1}$, pick an
  element $c_i$ of $J$ in the set, if $J$ intersects $A_i \setminus A_{i-1}$. Since $A_i$'s are
  closed, that means the subset $c_i$ has the discrete topology. Furthermore, since points are
  closed, the set $\{\cup c_i\}$ is a closed subset of $J$, hence compact. And compact spaces with
  discrete topology are finite. That means only finitely many $A_i \setminus A_{i-1}$ intersect $J$.
  This means the map factors through at some finite stage.
\end{proof}

\begin{thm}[Cofibre sequence]
  Put in result
\end{thm}

\begin{thm}[Alternative characterization of $\lim^1$]
  If $F$ is an object in the tower category, then $\lim^1(F)$ is the cokernel of the following map.
  \begin{align*}
    \alpha_F : \prod_{i \in \mathbb{N}} F_i &\to \prod_{i \in \mathbb{N}} F_i \\
    \alpha_F : (g_0, g_1, g_2, \ldots) &\mapsto \left(g_0 - f_1(g_1), g_1 - f_2(g_2), \ldots \right)
  \end{align*}
\end{thm}

\begin{proof}
  fill in later
\end{proof}

\begin{thm}[Milnor exact sequence] \label{thm-milnor} If $\{ i_n : X_n \hookrightarrow X_{n+1} \}$
  for $n \in \mathbb{N}$ are a sequence of nested CW subcomplexes such that $X = \bigcup_n X_n$, and
  $h^{\ast}$ is a reduced cohomology theory, then we have the following exact sequence for all
  $i \geq 1$.
  \begin{align*}
    0 \rightarrow {\lim_{n}}^1 h^{i-1}(X_n) \rightarrow h^i(X) \rightarrow \lim_n h^i(X_n) \rightarrow 0
  \end{align*}
\end{thm}

\begin{proof}
  fill in later
\end{proof}

% \bibliography{references} \bibliographystyle{amsplain}

\end{document}
