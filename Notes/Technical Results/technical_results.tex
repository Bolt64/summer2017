\documentclass[12pt, notitlepage]{article}
\usepackage[utf8]{inputenc}
\usepackage{graphicx}
\graphicspath{ {images/} }

\usepackage[english]{babel}
\usepackage[nottoc]{tocbibind}

\usepackage{hyperref}
\usepackage[left=3cm,right=3cm,top=2cm,bottom=2cm]{geometry}

\usepackage{bbm}

\usepackage{amsmath}
\usepackage{amsthm}
\usepackage{amsfonts}
\usepackage{amssymb}
\newtheorem{thm}{Theorem}[section]
\newtheorem{lem}[thm]{Lemma}
\newtheorem{prop}[thm]{Proposition}
\newtheorem{cor}[thm]{Corollary}
\newtheorem{conj}[thm]{Conjecture}
\newtheorem{exmp}[thm]{Example}
\newtheorem{remark}[thm]{Remark}

\usepackage{mathtools}
\usepackage{tikz-cd}

\usepackage{xcolor}

\theoremstyle{definition}
\newtheorem{defn}{Definition}[section]

\title{Proofs of some fairly technical results}
\author{Sayantan Khan}
\date{July 2017}

\usepackage{gentium}

\newcommand{\cat}[1]{\mathrm{#1}}
\newcommand{\cohomtheorie}{{h}^{\ast}}
\newcommand{\calz}{\mathcal{Z}}

\makeatletter
\newcommand{\colim@}[2]{%
  \vtop{\m@th\ialign{##\cr
    \hfil$#1\operator@font colim$\hfil\cr
    \noalign{\nointerlineskip\kern1.5\ex@}#2\cr
    \noalign{\nointerlineskip\kern-\ex@}\cr}}%
}
\newcommand{\colim}{%
  \mathop{\mathpalette\colim@{\rightarrowfill@\textstyle}}\nmlimits@
}
\makeatother

\begin{document}
\maketitle

\tableofcontents

\newpage

\section{Comparison theorem for cohomology theories}
\label{sec:comp-theor-cohom}

Fill in later

\section{Brown's representability theorem}
\label{sec:browns-repr-theor}

In this section, we shall see that all reduced cohomology theories that satisfy the wedge sum
($\mathrm{DV}$) axiom are representable functors, i.e. they are naturally isomorphic to the hom
functor in the homotopy category $\cat{hCW}_{\ast}$. In particular, for a given reduced cohomology
theory $\cohomtheorie$, we'll construct a sequence of spaces $\calz(n)$, which we'll call a
spectrum, such that $h^n(X)$ is naturally isomorphic to $\left[X, \calz(n)\right]$.

\subsection{Spectra and cohomology theories}
\label{sec:spectra-cohom-theor}

\begin{defn}[Spectrum]
  A spectrum is a $\mathbb{Z}$ indexed sequence of pointed spaces $\calz(n)$ together with structure
  maps $\sigma_n: \Sigma \calz(n) \to \calz(n+1)$. If the adjoints of the structure maps, i.e. the
  maps $\widetilde{\sigma}_n: \calz(n) \to \Omega \calz(n+1)$ are homotopy equivalences, then the
  spectrum is called an $\Omega$-spectrum.
\end{defn}

\begin{prop}
  Given a spectrum $\calz$, one can define the following functor.
  \begin{align*}
    h^n(X; \calz) = \colim_{k \to \infty} \left[S^k \wedge X, \calz(k+n)\right]
  \end{align*}
  This is a contravariant functor which satisfies the homotopy invariance $(\mathrm{H})$, suspension
  $(\mathrm{S})$, and exactness $(\mathrm{E})$ axiom. Furthermore, if $\calz$ is an
  $\Omega$-spectrum, the functor also satisfies the $(\mathrm{DV})$ axiom. It is therefore a reduced
  cohomology theory.
\end{prop}

\begin{proof}
  We'll deal with the axioms one at a time.
  \begin{description}
  \item[Homotopy invariance (H):] This is obvious, because we are working in the homotopy category
    $\cat{hCW}_{\ast}$.  Any two continuous maps which are homotopy equivalent are the same morphism
    in this category, hence correspond to the same map to $\calz(k+n)$. And since the maps are same
    for all $k$, they also agree in the colimit.
  \item[Suspension (S):] We need to show there is a natural isomorphism from $h^n(X)$ to
    $h^{n+1}(\Sigma X)$. To see this natural isomorphism, note that $h^n(X)$ is the colimit of the
    following diagram.
    \begin{align*}
      [X, \calz(n)] \to [\Sigma X, \calz(n+1)] \to [\Sigma^2 X, \calz(n+2)] \to \cdots
    \end{align*}
    But for a diagram like this, the colimit won't change if we drop finitely many groups from the
    beginning of the diagram. Hence, $h^n(X)$ is also the colimit of the following diagram.
    \begin{align*}
      [\Sigma X, \calz(n+1)] \to [\Sigma^2 X, \calz(n+2)] \to [\Sigma^3 X, \calz(n+3)] \to \cdots
    \end{align*}
    
    Similarly, $h^{n+1}(\Sigma X)$ is the colimit of the following diagram.
    \begin{align*}
      [\Sigma X, \calz(n+1)] \to [\Sigma^2 X, \calz(n+2)] \to [\Sigma^3 X, \calz(n+3)] \to \cdots
    \end{align*}
    But now we notice that the groups are the same in the diagram for $h^n(X)$ and
    $h^{n+1}(\Sigma X)$. We draw isomorphisms between the corresponding groups, and that gives an
    natural isomorphism in the colimit.
  \item[Exactness (E):] We need to show for any cofibration $i: A \hookrightarrow X$, the following
    sequence is exact.
    \begin{align*}
      h^n(A) \leftarrow h^n(X) \leftarrow h^n\left( X/A \right)
    \end{align*}
    Using the cofibre sequence, we get that following sequences are exact for all $k$.
    \begin{align*}
      [\Sigma^k A, \calz(n+k)] \leftarrow [\Sigma^k X, \calz(n+k)] \leftarrow \left[\Sigma^k \left(X/A\right), \calz(n+k) \right]
    \end{align*}
    And since filtered colimits of exact sequences are exact in $\cat{AbGrp}$, the cohomology
    sequence is exact. ({\color{red} Should check this by hand in this particular case.})
  \item[Wedge sum (DV):] Using the fact that suspension and loop space functors are adjoints, we
    have the following.
    \begin{align*}
      [\Sigma^k X, \calz(n+k)] \cong [X, \Omega^k \calz(n+k)]
    \end{align*}
    This means that $h^n(X)$ is the colimit of the following diagram.
    \begin{align*}
      [X, \calz(n)] \rightarrow [X, \Omega \calz(n+1)] \rightarrow [X, \Omega^2 \calz(n+2)] \rightarrow \cdots
    \end{align*}
    If $\calz$ is an $\Omega$-spectrum, then all the arrows in the above diagram are
    isomorphisms. That means $h^n(X) = [X, \calz(n)]$. But the functor $[\cdot, \calz(n)]$ satisfies
    $(\mathrm{DV})$ axiom. That means $h^{\ast}$ satisfies $(\mathrm{DV})$ axiom.
  \end{description}
\end{proof}

\newpage

\appendix

\section{Definitions and notation}
\label{sec:definitions-notation}

\begin{defn}[Suspension of a pointed space]
  The suspension $\Sigma X$ of a pointed space $X$ is the smash product $S^1 \wedge X$.
\end{defn}

\begin{defn}[Loop space of a pointed space]
  The loop spaces $\Omega X$ of a pointed space $X$ is the set of all pointed maps from $S^1$ to $X$
  with the compact-open topology.
\end{defn}

\section{Some useful lemmas and theorems}
\label{sec:some-useful-lemmas}

\textbf{Note:} Although we state many of the lemmas here for $\cat{TOP}$, they are also true for
$\cat{TOP}_{\ast}$, and the proof is similar.

\begin{lem}
  If $i: A \hookrightarrow X$ is a cofibration (in the category $\cat{TOP}$), then the mapping cone
  $C(i)$ is homotopy equivalent to $X/A$.
\end{lem}

\begin{proof}
  We will first construct the maps to and from $C(i)$ to $X/A$. The maps from $C(i)$ to $X/A$ is the
  map that collapses the cone of $A$ to a point corresponding to $A$ in $X/A$. Now consider a map
  from $H: A \times I$ to $C(i)$, such that $H$ contracts $A$ to a point in $C(i)$, starting from
  the inclusion of $A$ in $X$. Let the map $j$ from $X$ to $C(i)$ be the inclusion map. Since $i$ is
  a cofibration, we can extend $H$ with the initial condition $j$ to a map $J: X \times I \to
  C(i)$. But $J(\cdot, 1)$ collapses $A$ to a point. That means it factors through a $X/A$. This
  gives us a map $k$ from $X/A$ to $C(i)$.

  The fact that these maps are homotopy inverses can be verified using the homotopy
  $J$. ({\color{red} Not sure of this. Verify later.})
\end{proof}

\begin{lem}
  In the category $\cat{TOP}$, the following sequence is h-coexact.
  \begin{align*}
    A \xrightarrow{f} B \xrightarrow{i} C(f)
  \end{align*}
  That means for any space $Z$, the following sequence of abelian groups is exact.
  \begin{align*}
    [A, Z] \leftarrow [B,Z] \leftarrow [C(f), Z]
  \end{align*}
\end{lem}

\begin{proof}
  If an element $[c] \in [B,Z]$ goes to $0$ in $[A, Z]$, that means $c \circ f: A \to Z$ is
  nullhomotopic, where $c$ is a representative of $[c]$. But that means there is some function
  $d \in C(f)$ such that $c = d \circ i$. This shows the exactness of the sequence.
\end{proof}

\begin{lem}
  If $K$ is a compact space, let $A_i$ be a sequence of spaces where points are closed,
  and $A$ is the colimit of the following diagram:
  \begin{align*}
    A_0 \hookrightarrow A_1 \hookrightarrow A_2 \hookrightarrow \cdots
  \end{align*}
  where all the embeddings are closed, then a map from $K$ to $A$ factors finitely through some
  $A_i$.
\end{lem}

\begin{proof}
  Let $J = f(K)$ be the compact image of $K$ in $A$. For each set $A_i \setminus A_{i-1}$, pick
  an element $c_i$ of $J$ in the set, if $J$ intersects $A_i \setminus A_{i-1}$. Since $A_i$'s
  are closed, that means the subset $c_i$ has the discrete topology. Furthermore, since points are
  closed, the set $\{\cup c_i\}$ is a closed subset of $J$, hence compact. And compact spaces with
  discrete topology are finite. That means only finitely many $A_i \setminus A_{i-1}$ intersect $J$.
  This means the map factors through at some finite stage.
\end{proof}

\begin{thm}[Cofibre sequence]
  Put in result
\end{thm}

% \bibliography{references} \bibliographystyle{amsplain}

\end{document}
