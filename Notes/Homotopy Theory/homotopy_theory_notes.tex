\documentclass[12pt, notitlepage]{article}
\usepackage[utf8]{inputenc}
\usepackage{graphicx}
\graphicspath{ {images/} }

\usepackage[english]{babel}
\usepackage[nottoc]{tocbibind}

\usepackage{hyperref}
\usepackage[left=3cm,right=3cm,top=2cm,bottom=2cm]{geometry}

\usepackage{bbm}

\usepackage{amsmath}
\usepackage{amsthm}
\usepackage{amsfonts}
\usepackage{amssymb}
\newtheorem{thm}{Theorem}[section]
\newtheorem{lem}[thm]{Lemma}
\newtheorem{prop}[thm]{Proposition}
\newtheorem{cor}[thm]{Corollary}
\newtheorem{conj}[thm]{Conjecture}
\newtheorem{exmp}[thm]{Example}

\usepackage{mathtools}
\usepackage{tikz-cd}

\usepackage{xcolor}

\theoremstyle{definition}
\newtheorem{defn}{Definition}[section]

\title{Notes on Homotopy Theory}
\author{Sayantan Khan}
\date{July 2017}

\usepackage{gentium}

\newcommand{\cat}[1]{\mathrm{#1}}

\begin{document}
\maketitle

\tableofcontents

\newpage

\section{Categorical preliminaries}
In this section, we'll define the categories we'll be dealing with in the rest of the notes. We'll also define some categorical constructions: in particular the \emph{pushout} and the \emph{pullback}.

\subsection{Some important categories}
\begin{description}
    \item[$\cat{SET}$:] This is the category of sets, where the objects are sets, and the morphisms between objects are set maps.
    \item[$\cat{TOP}$:] This is the category of topological spaces, where the objects are topological spaces, and the maps are continuous maps between topological spaces.
    \item[$\cat{hTOP}$:] This is the category with the objects being topological spaces, but the maps are homotopy classes of continuous maps, rather than being continuous maps themselves.
    \item[$\cat{TOP^0}$:] This is the category of pointed spaces, i.e. the objects are tuples of spaces and a basepoint in them, and morphisms are continuous maps that take basepoints to basepoints.
    \item[$\cat{hTOP^0}$:] This is the homotopy category of pointed spaces, i.e. the objects are the same as in $\cat{TOP^0}$, but the maps are homotopy classes of maps between pointed spaces.
    \item[$\cat{TOP(2)}$:] This is the category of pairs of spaces. The objects here are $(X,A)$, where $A \subset X$, and a morphism from $(X, A)$ to $(Y, B)$ is a continuous map $f: X \to Y$ such that $f(A) \subset B$.
    \item[$W(X,Y)$:] Here, $X$ and $Y$ are two topological spaces. The objects of $W(X,Y)$ are the continuous maps between $X$ and $Y$, and the morphisms are homotopies between maps.
    \item[$\cat{TOP}_B$:] Given a fixed topological space $B$, an object in the category $\cat{TOP}_B$ is a topological space $X$ along with a map $f: X \to B$. Given two objects $(X, f : X \to B)$ and $(Y, g : Y \to B)$, a morphism from the former to the latter is a continuous map $h$ from $X$ to $Y$ such that the following diagram commutes.
    \[
    \begin{tikzcd}
    X \arrow{r}{f} \arrow[swap]{dr}{h} & B  \\
    & Y \arrow{u}{g}
    \end{tikzcd}
    \]
    This is the \emph{category of spaces over $B$}.
    \item[$\cat{hTOP}_B$:] This is the homotopy category of $\cat{TOP}_B$, where the objects are the same, but the maps are quotiented out by homotopies.
    \item[$\cat{TOP}^A$:] Given a fixed topological space $A$, an object in the category $\cat{TOP}^A$ is a space $X$ along with a map $f: A \to X$. Given two objects $(X, f: A \to X)$ and $(Y, g: A \to Y)$, a morphism between these objects is a map $h: X \to Y$ such that the following diagram commutes.
    \[
    \begin{tikzcd}
    A \arrow{r}{f} \arrow[swap]{d}{g} & X \ar{dl}{h}  \\
    Y & \\
    \end{tikzcd}
    \]
    This is the \emph{category of spaces under $A$}.
    \item[$\cat{hTOP}^A$:] This is the homotopy category of $\cat{TOP}^A$, described in a manner similar to $\cat{hTOP}_B$.
\end{description}

\subsection{Categorical constructions}
\subsubsection{Product}

\begin{defn}
Given two objects $A$ and $B$ in a category $\mathcal{C}$, their product is an object $A \times B$ along with maps $\pi_1 : A \times B \to A$ and $\pi_2 : A \times B \to B$ such that for any object $F$ with maps $f_1 : F \to A$ and $f_2: F \to B$, there exists a unique map from $F$ to $A \times B$ making the following diagram commute.
\[
\begin{tikzcd}
&      F \arrow[swap]{dl}{f_1} \arrow{dr}{f_2} \arrow[dashed]{d}{\exists !}     & \\
A & A \times B \arrow{l}{\pi_1} \arrow[swap]{r}{\pi_2}  & B \\
\end{tikzcd}
\]    
\end{defn}

Products may not exist in all categories, but when they do, they are unique. They exist in $\cat{SET}$ and $\cat{TOP}$, are the usual product.

\subsubsection{Coproduct}

\begin{defn}
    In a category $\mathcal{C}$, the coproduct of objects $A$ and $B$ is the object $A \coprod B$ along with maps $i_1: A \to A \coprod B$ and $i_2: B \to A \coprod B$ such that for any pair of maps $g_1 : A \to G$ and $g_2: B \to G$, there exists a unique factorization via $A \coprod B$.
    \[
    \begin{tikzcd}
    A \ar{r}{i_1} \ar[swap]{dr}{g_1} & A \coprod B \ar[dashed]{d}{\exists !} & B \ar[swap]{l}{i_2} \ar{dl}{g_2} \\
    & G & \\
    \end{tikzcd}
    \]
    
\end{defn}

Coproducts exists in $\cat{SET}$ and $\cat{TOP}$ and are the disjoint union in these two categories. In $\cat{TOP^0}$, the coproduct is the wedge sum along the basepoint.

\subsubsection{Pullback}

\begin{defn}
    In a category $\mathcal{C}$, given two maps $f : X \to B$ and $g: Y \to B$, the pullback of $f$ and $g$ is the following diagram
    \[
    \begin{tikzcd}
    W \ar{r}{F} \ar[swap]{d}{G} & Y \ar{d}{g} \\
    X \ar[swap]{r}{f} & B \\
    \end{tikzcd}
    \]
    along with the universal property that for any $V$ with maps $F_V$ and $G_V$ to $X$ and $Y$, $F_V$ and $G_V$ factor uniquely through $W$.
    \[
    \begin{tikzcd}
    V \ar[bend left]{drr}{F_V} \ar[bend right, swap]{ddr}{G_V} \ar[dashed]{dr}{\exists !} & & \\
    & W \ar{r}{F} \ar[swap]{d}{G} & Y \ar{d}{g} \\
    &  X \ar[swap]{r}{f} & B \\
    \end{tikzcd}
    \]
\end{defn}
In $\cat{TOP}$, the pullback exists, and is given by the following subspace.
\[
W = \{ (x,y) \in X \times Y\ |\ f(x) = g(y) \}
\]
Alternatively, a pullback can be shown to be the product in the category $\cat{TOP}_B$.

\subsubsection{Pushout}

\begin{defn}
    A pushout is the dual notion to a pullback. Given a category $\mathcal{C}$, and maps $f: A \to X$ and $g: A \to Y$, the pushout of $f$ and $g$ is the following diagram.
    \[
    \begin{tikzcd}
    A \ar{r}{f} \ar[swap]{d}{g} & X \ar{d}{G} \\
    Y \ar[swap]{r}{F} & W \\
    \end{tikzcd}
    \]
    $W$ must also satisfy the following universal property.
    \[
    \begin{tikzcd}
    A \ar{r}{f} \ar[swap]{d}{g} & X \ar{d}{G} \ar[bend left]{ddr}{G_V} & \\
    Y \ar[swap]{r}{F} \ar[swap, bend right]{drr}{F_V} & W \ar[dashed]{dr}{\exists !} & \\
    & & V \\
    \end{tikzcd}
    \]
\end{defn}
In $\cat{TOP}$, the pushout $W$ is the following space.
\[
W = \frac{\left(X \coprod Y\right)}{f(a) \sim g(a)}
\]
Alternatively, a pushout can be seen as a coproduct in the category $\cat{TOP}^A$.


\section{Homotopical Constructions}
In this section, we'll cover the construction of the essential spaces in homotopy theory: the mapping cylinder, cones, suspensions, and loop spaces; we'll cover the pointed and unpointed versions of these, and prove their homotopy equivalence whenever applicable.

\subsection{Mapping cylinder}
\begin{defn}
    Given a map $f: X \to Y$, the mapping cylinder $Z(f)$ is constructed via the following pushout.
    \[
    \begin{tikzcd}
    X \ar{r}{f} \ar[swap]{d}{i_1^X} & Y \ar{d}{J} \\
    X \times I \ar[swap]{r}{a} & Z(f) \\
    \end{tikzcd}
    \]
\end{defn}
Topologically, the mapping cylinder is the disjoint union of $X \times I$ and $Y$ quotiented with the relation $(x, 1) \sim f(x)$.
\[
\text{\color{red} Insert picture later}
\]

%\[
%\begin{tikzcd}[arrows=to]
%\cdots \rar & H_n(A \cap B) \rar & H_n(A) \oplus H_n(B) \rar & H_n(X) \rar & \hphantom{0}\\
%\hphantom{\cdots} \rar 
%& H_{n-1}(A \cap B) \rar 
%& \makebox[\widthof{$H_n(A) \oplus H_n(B)$}][c]{$\cdots\hfill \cdots$} \rar
%&  H_0(X) \rar & 0
%\end{tikzcd}
%\]


%\bibliography{references}
%\bibliographystyle{amsplain}


\end{document}
