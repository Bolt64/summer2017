\documentclass[12pt, notitlepage]{article}
\usepackage[utf8]{inputenc}
\usepackage{graphicx}
\graphicspath{ {images/} }

\usepackage[english]{babel}
\usepackage[nottoc]{tocbibind}

\usepackage{hyperref}
\usepackage[left=3cm,right=3cm,top=2cm,bottom=2cm]{geometry}

\usepackage{bbm}

\usepackage{amsmath}
\usepackage{amsthm}
\usepackage{amsfonts}
\usepackage{amssymb}
\newtheorem{thm}{Theorem}[section]
\newtheorem{lem}[thm]{Lemma}
\newtheorem{prop}[thm]{Proposition}
\newtheorem{cor}[thm]{Corollary}
\newtheorem{conj}[thm]{Conjecture}
\newtheorem{exmp}[thm]{Example}

\usepackage{mathtools}
\usepackage{tikz-cd}

\usepackage{xcolor}

\theoremstyle{definition}
\newtheorem{defn}{Definition}[section]

\title{Notes on Homotopy Theory}
\author{Sayantan Khan}
\date{July 2017}

\usepackage{gentium}

\newcommand{\cat}[1]{\mathrm{#1}}
\newcommand{\id}{\mathrm{id}}
\newcommand{\homtheorie}{{h}_{\ast}}
\newcommand{\redhom}{\widetilde{h}_{\ast}}

\begin{document}
\maketitle

\tableofcontents

\newpage

\section{Categorical preliminaries}
In this section, we'll define the categories we'll be dealing with in the rest of the notes.  We'll
also define some categorical constructions: in particular the \emph{pushout} and the
\emph{pullback}.

\subsection{Some important categories}
\begin{description}
\item[$\cat{SET}$:] This is the category of sets, where the objects are sets, and the morphisms
  between objects are set maps.
\item[$\cat{TOP}$:] This is the category of topological spaces, where the objects are topological
  spaces, and the maps are continuous maps between topological spaces.
\item[$\cat{hTOP}$:] This is the category with the objects being topological spaces, but the maps
  are homotopy classes of continuous maps, rather than being continuous maps themselves.
\item[$\cat{TOP^0}$:] This is the category of pointed spaces, i.e. the objects are tuples of spaces
  and a basepoint in them, and morphisms are continuous maps that take basepoints to basepoints.
\item[$\cat{hTOP^0}$:] This is the homotopy category of pointed spaces, i.e.  the objects are the
  same as in $\cat{TOP^0}$, but the maps are homotopy classes of maps between pointed spaces.
\item[$\cat{TOP(2)}$:] This is the category of pairs of spaces. The objects here are $(X,A)$, where
  $A \subset X$, and a morphism from $(X, A)$ to $(Y, B)$ is a continuous map $f: X \to Y$ such that
  $f(A) \subset B$.
\item[$W(X,Y)$:] Here, $X$ and $Y$ are two topological spaces. The objects of $W(X,Y)$ are the
  continuous maps between $X$ and $Y$, and the morphisms are homotopies between maps.
\item[$\cat{TOP}_B$:] Given a fixed topological space $B$, an object in the category $\cat{TOP}_B$
  is a topological space $X$ along with a map $f: X \to B$.  Given two objects $(X, f : X \to B)$
  and $(Y, g : Y \to B)$, a morphism from the former to the latter is a continuous map $h$ from $X$
  to $Y$ such that the following diagram commutes.
  \[
    \begin{tikzcd}
      X \arrow{r}{f} \arrow[swap]{dr}{h} & B  \\
      & Y \arrow{u}{g}
    \end{tikzcd}
  \]
  This is the \emph{category of spaces over $B$}.
\item[$\cat{hTOP}_B$:] This is the homotopy category of $\cat{TOP}_B$, where the objects are the
  same, but the maps are quotiented out by homotopies.
\item[$\cat{TOP}^A$:] Given a fixed topological space $A$, an object in the category $\cat{TOP}^A$
  is a space $X$ along with a map $f: A \to X$. Given two objects $(X, f: A \to X)$ and
  $(Y, g: A \to Y)$, a morphism between these objects is a map $h: X \to Y$ such that the following
  diagram commutes.
  \[
    \begin{tikzcd}
      A \arrow{r}{f} \arrow[swap]{d}{g} & X \ar{dl}{h}  \\
      Y & \\
    \end{tikzcd}
  \]
  This is the \emph{category of spaces under $A$}.
\item[$\cat{hTOP}^A$:] This is the homotopy category of $\cat{TOP}^A$, described in a manner similar
  to $\cat{hTOP}_B$.
\end{description}

\subsection{Categorical constructions}
\subsubsection{Product}

\begin{defn}
  Given two objects $A$ and $B$ in a category $\mathcal{C}$, their product is an object $A \times B$
  along with maps $\pi_1 : A \times B \to A$ and $\pi_2 : A \times B \to B$ such that for any object
  $F$ with maps $f_1 : F \to A$ and $f_2: F \to B$, there exists a unique map from $F$ to
  $A \times B$ making the following diagram commute.
  \[
    \begin{tikzcd}
      &      F \arrow[swap]{dl}{f_1} \arrow{dr}{f_2} \arrow[dashed]{d}{\exists !}     & \\
      A & A \times B \arrow{l}{\pi_1} \arrow[swap]{r}{\pi_2}  & B \\
    \end{tikzcd}
  \]
\end{defn}

Products may not exist in all categories, but when they do, they are unique.  They exist in
$\cat{SET}$ and $\cat{TOP}$, are the usual cartesian product.

\subsubsection{Coproduct}

\begin{defn}
  In a category $\mathcal{C}$, the coproduct of objects $A$ and $B$ is the object $A \coprod B$
  along with maps $i_1: A \to A \coprod B$ and $i_2: B \to A \coprod B$ such that for any pair of
  maps $g_1 : A \to G$ and $g_2: B \to G$, there exists a unique factorization via $A \coprod B$.
  \[
    \begin{tikzcd}
      A \ar{r}{i_1} \ar[swap]{dr}{g_1} & A \coprod B \ar[dashed]{d}{\exists !} & B \ar[swap]{l}{i_2} \ar{dl}{g_2} \\
      & G & \\
    \end{tikzcd}
  \]
\end{defn}

Coproducts exists in $\cat{SET}$ and $\cat{TOP}$ and are the disjoint union in these two
categories. In $\cat{TOP^0}$, the coproduct is the wedge sum along the basepoint.

\subsubsection{Pullback}

\begin{defn}
  In a category $\mathcal{C}$, given two maps $f : X \to B$ and $g: Y \to B$, the pullback of $f$
  and $g$ is the following diagram
  \[
    \begin{tikzcd}
      W \ar{r}{F} \ar[swap]{d}{G} & Y \ar{d}{g} \\
      X \ar[swap]{r}{f} & B \\
    \end{tikzcd}
  \]
  along with the universal property that for any $V$ with maps $F_V$ and $G_V$ to $X$ and $Y$, $F_V$
  and $G_V$ factor uniquely through $W$.
  \[
    \begin{tikzcd}
      V \ar[bend left]{drr}{F_V} \ar[bend right, swap]{ddr}{G_V} \ar[dashed]{dr}{\exists !} & & \\
      & W \ar{r}{F} \ar[swap]{d}{G} & Y \ar{d}{g} \\
      &  X \ar[swap]{r}{f} & B \\
    \end{tikzcd}
  \]
\end{defn}
In $\cat{TOP}$, the pullback exists, and is given by the following subspace.
\[
  W = \{ (x,y) \in X \times Y\ |\ f(x) = g(y) \}
\]
Alternatively, a pullback can be shown to be the product in the category $\cat{TOP}_B$.

\subsubsection{Pushout}

\begin{defn}
  A pushout is the dual notion to a pullback. Given a category $\mathcal{C}$, and maps $f: A \to X$
  and $g: A \to Y$, the pushout of $f$ and $g$ is the following diagram.
  \[
    \begin{tikzcd}
      A \ar{r}{f} \ar[swap]{d}{g} & X \ar{d}{G} \\
      Y \ar[swap]{r}{F} & W \\
    \end{tikzcd}
  \]
  $W$ must also satisfy the following universal property.
  \[
    \begin{tikzcd}
      A \ar{r}{f} \ar[swap]{d}{g} & X \ar{d}{G} \ar[bend left]{ddr}{G_V} & \\
      Y \ar[swap]{r}{F} \ar[swap, bend right]{drr}{F_V} & W \ar[dashed]{dr}{\exists !} & \\
      & & V \\
    \end{tikzcd}
  \]
\end{defn}
In $\cat{TOP}$, the pushout $W$ is the following space.
\[
  W = \frac{\left(X \coprod Y\right)}{f(a) \sim g(a)}
\]
Alternatively, a pushout can be seen as a coproduct in the category $\cat{TOP}^A$.


\section{Homotopical Constructions}
In this section, we'll cover the construction of the essential groups and spaces in homotopy theory:
the homotopy groupoid, mapping cylinder, cones, suspensions, and loop spaces.

\subsection{Homotopy groupoid}
\begin{defn}
  Let $X$ and $Y$ be topological spaces. The category $\Pi(X,Y)$ has its objects as maps from $X$ to
  $Y$, and its morphisms are homotopies between maps quotiented by the following relation. Two
  homotopies between maps $f$ and $g$, $\mathcal{P}$ and $\mathcal{Q}$ are the same morphism if
  there is a homotopy $\mathcal{M}$ from $\mathcal{P}$ to $\mathcal{Q}$ relative to \footnote{A
    homotopy relative to a subspace is a homotopy that is constant on that subspace.}
  $X \times \partial I$.
\end{defn}
The quotienting gives the collection of morphisms a groupoid structure. In particular, associativity
only works out because of the quotienting. The fundamental groupoid is a special case of the
homotopy groupoid $\Pi(X,Y)$, when $X$ is just a point. Similarly, we can describe the pointed
version of the homotopy groupoid, which we denote by $\Pi^0(X,Y)$ for pointed spaces $X$ and $Y$.

\subsection{Mapping cylinder}
\begin{defn}
  Given a map $f: X \to Y$, the mapping cylinder $Z(f)$ is constructed via the following pushout.
  \[
    \begin{tikzcd}
      X \ar{r}{f} \ar[swap]{d}{i_1^X} & Y \ar{d}{J} \\
      X \times I \ar[swap]{r}{a} & Z(f) \\
    \end{tikzcd}
  \]
\end{defn}
Topologically, the mapping cylinder is the disjoint union of $X \times I$ and $Y$ quotiented with
the relation $(x, 1) \sim f(x)$.
\begin{figure}[h]
  \centering \includegraphics[scale=0.25]{mapping_cylinder.png}
  \caption{The mapping cylinder ({\color{red}Temporary. Put citation.})}
\end{figure}

We construct some more maps.
\begin{align*}
  q: Z(&f) \to Y \\
  q(x,t) &:= f(x) \\
  q(y) &:= y  
\end{align*}
\begin{align*}
  j: X &\to Z(f) \\
  j(x) &:= (x, 0)
\end{align*}
We now have the following relations.
\begin{align*}
  qj &= f \\
  qJ &= \id_{Y}
\end{align*}
We can also see the map $Jq$ is homotopic to $\id_{Z(f)}$ relative to the $Y$ subspace.  This means
$Z(f)$ is homotopy equivalent to $Y$ and $q$ and $J$ are the homotopy equivalence.  Note that $j$ is
a closed embedding. We have thus decomposed $f$ into a closed embedding $j$, and a homotopy
equivalence $q$.

\subsection{Suspension}
\begin{defn}
  content...
\end{defn}

\section{Generalized (Co)homology Theories}
\label{sec:gener-cohom-theor}

\subsection{Correspondence between unreduced and reduced homology theory}
\label{sec:corr-betw-homol}

\begin{defn}[Unreduced homology]
  An unreduced homology theory is a functor $h_{\ast}$ from the category $\cat{hCW}^2$ to the category of
  $\mathbb{Z}$ graded abelian groups $\mathbb{Z}-\cat{Ab}$ such that $h_{\ast}$ along with the following
  natural transformations.
  \begin{align*}
    \partial_{\ast} : h_{\ast} \to h_{\ast - 1} \circ I
  \end{align*}
  Here, $I$ sends $(X, A)$ to $(A, \varnothing)$. The funtor must satisfy the following axioms.
  \begin{description}
  \item[Homotopy equivalence (H):] If two spaces in $\cat{hCW}^2$ are homotopy equivalent via a map of pairs
    $f$, then $h_{\ast}(f)$ is a natural isomorphism.
  \item[Exactness (E):] We have the following long exact sequence. 
    \[
      \begin{tikzcd}
        \cdots \ar{r}{} & h_{n}(A) \ar{r}{i_{\ast}} & h_{n}(X) \ar{r}{j_{\ast}} & h_{n}(X,A) \ar{r}{\partial_{\ast}} & h_{n-1}(A) \ar{r} & \cdots \\
      \end{tikzcd}
    \]
  \item[Excision (A):] For subcomplexes $A$ and $B$, the map $i : (A, A \cap B) \hookrightarrow (A \cup B, B)$ induces an isomorphism from
    $h_{\ast}(A, A \cap B) \to h_{\ast} (A \cup B, B)$.
  \item[Direct union (DV):] For any indexing set $\Omega$, and a space $X = \coprod_{\alpha \in \Omega} X_{\alpha}$, the induced maps
    $h_{\ast}(X_{\alpha}) \to h_{\ast}(X)$ induce the following isomorphism.
    \begin{align*}
      \bigoplus_{\alpha \in \Omega} h_{\ast}(X_{\alpha}) \to h_{\ast}(X)
    \end{align*}
  \end{description}
\end{defn}

\begin{defn}[Reduced homology]
  A reduced homology theory is a functor $\redhom$ from $\cat{hCW}^0$ to $\mathbb{Z}$ graded abelian groups
  along with the following natural transformation $s_{\ast}$.
  \begin{align*}
    \redhom(X) \to \widetilde{h}_{\ast + 1} (SX)
  \end{align*}
  This functor satisfies the following axioms.
  \begin{description}
  \item[Homotopy invariance (H):] A pointed homotopy between spaces induces isomorphisms in the homology groups.
  \item[Suspension (S):] The map $s_{\ast}$ is a natural isomorphism.
  \item[Exactness (E):] If $i: A \hookrightarrow X$ is a cofibration, then the following sequence is exact.
    \begin{align*}
      \redhom(A) \rightarrow \redhom(X) \rightarrow \redhom\left( X/A \right)
    \end{align*}
  \item[Direct union (DV):] If $X = \bigvee_{\alpha} X_{\alpha}$, then the inclusions induce the following isomorphism.
    \begin{align*}
      \bigoplus_{\alpha} \redhom(X_{\alpha}) \to \redhom(X)
    \end{align*}
  \end{description}
\end{defn}

Unreduced and reduced cohomology are defined in a similar manner, by using contravariant functors and reversing all the arrows.

% \bibliography{references} \bibliographystyle{amsplain}

\end{document}
