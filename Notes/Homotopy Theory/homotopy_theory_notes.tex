\documentclass[12pt, notitlepage]{article}
\usepackage[utf8]{inputenc}
\usepackage{graphicx}
\graphicspath{ {images/} }

\usepackage[english]{babel}
\usepackage[nottoc]{tocbibind}

\usepackage{hyperref}
\usepackage[left=3cm,right=3cm,top=2cm,bottom=2cm]{geometry}

\usepackage{bbm}

\usepackage{amsmath}
\usepackage{amsthm}
\usepackage{amsfonts}
\usepackage{amssymb}
\newtheorem{thm}{Theorem}[section]
\newtheorem{lem}[thm]{Lemma}
\newtheorem{prop}[thm]{Proposition}
\newtheorem{cor}[thm]{Corollary}
\newtheorem{conj}[thm]{Conjecture}
\newtheorem{exmp}[thm]{Example}

\usepackage{mathtools}
\usepackage{tikz-cd}

\theoremstyle{definition}
\newtheorem{defn}{Definition}[section]

\title{Notes on Homotopy Theory}
\author{Sayantan Khan}
\date{July 2017}

\usepackage{gentium}

\begin{document}
\maketitle

\tableofcontents

\newpage

\[
\begin{tikzcd}[arrows=to]
\cdots \rar & H_n(A \cap B) \rar & H_n(A) \oplus H_n(B) \rar & H_n(X) \rar & \hphantom{0}\\
\hphantom{\cdots} \rar 
& H_{n-1}(A \cap B) \rar 
& \makebox[\widthof{$H_n(A) \oplus H_n(B)$}][c]{$\cdots\hfill \cdots$} \rar
&  H_0(X) \rar & 0
\end{tikzcd}
\]


\bibliography{references}
\bibliographystyle{amsplain}


\end{document}
